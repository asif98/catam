\documentclass[11pt]{article}
\usepackage{amsthm, amssymb, amsmath}
\usepackage[margin=1in,letterpaper]{geometry}
\usepackage[pdftex]{graphicx}
\begin{document}
\renewcommand{\labelenumi}{\roman{enumi}}




\section*{Question 1}

A program for finding the multiplicative inverse can be found on page \pageref{sec:program1} named \texttt{inverse.m}. This program determines inverse of $a$ where $1\leq a \leq p-1$ by testing with every integer $b$ in the range $[1,p-1]$ untill it finds one s.t $ab \equiv 1 \pmod p$.\\

But if we have that $b$ is the inverse of $a$, then we already know that $p-b$ is the inverse of $p-a$. So we can find the inverses of $a$ where $1\leq a \leq \frac{p-1}2$ by testing and then assign the inverse of both $a$ and $p-a$. This way we can speed up the procedure by a factor of $2$. A program for this can be found on page~\pageref{sec:program1} named \texttt{inverse2.m}.







\section*{Question 2}
% % % % % % % % % % % % % % % % % % % % % % % % % % % % % % % % % % % % % % % % % % % % % % % % % % % % % % % % % % % % % % % % % % % % % % % % % % % % % % % % % % % % % % % % % % % % % % % % % % % % % % % % % % % % % % % % % % % % % % % % % % % % % % % % % % % % % % % % % % % % % % % % % % % % % % % % % % % % % % % % % % % % % % % % % % % % % % % % % % % % % % % % % % % % % % % % %
For the program \texttt{inverse.m}, we have to do at most $(p-1)^2$ multiplications, $(p-1)^2$ division by $p$ and $(p-1)^2$ comparison(with $1$), so in total at most $3(p-1)^2$ steps. Hence the compexity is $p^2$.








\section*{Question 3}  A program for finding the row echelon form of a matrix can be found on page~\pageref{sec:prgram2} named \texttt{rowechelon.m}. We run the program for the given matrices $A_1$ and $A_2$.





\begin{verbatim}
>> A1 = [ 0 1 7 2 10 ; 8 0 2 5 1 ; 2 1 2 5 5 ; 7 4 5 3 0];
>> rowechelon(11,A1)

ans =

     1     0     3     0     0
     0     1     7     0     0
     0     0     0     1     0
     0     0     0     0     1

>> rowechelon(19,A1)

ans =

     1     0     0     0    13
     0     1     0     0     6
     0     0     1     0     3
     0     0     0     1     1

>> A2 = [ 6 16 11 14 1 4 ; 7 9 1 1 21 0 ; 8 2 9 12 17 7 ; 2 19 2 19 7 12 ];
>> rowechelon(23,A1)

ans =

     1     0     0    19    14
     0     1     0    22    19
     0     0     1     7     2
     0     0     0     0     0
\end{verbatim}

Easy to see that $A_1$ has rank $4$ in both cases of $\pmod 11$ and $19$ whereas $A_2$ has rank $3$. The rows of the row echelon form forms a basis for their row spaces. Hence bases for $A_1$ and $A_2$ can be given by
\begin{enumerate}
\item Basis for $A_1 \pmod{11}$ : $\{(1,0,3,0,0),(0,1,7,0,0),(0,0,0,1,0),(0,0,0,0,1)\}$
\item Basis for $A_1 \pmod{19}$ : $\{(1,0,0,0,12),(0,1,0,0,6),(0,0,1,0,3),(0,0,0,0,1,1)\}$
\item Basis for $A_2 \pmod{23}$ : $\{(1,0,0,19,24),(0,1,0,22,19),(0,0,1,7,2),(0,0,0,0,0)\}$
\end{enumerate}







\section*{Question 4} A program to compute a basis for the kernel of a matrix can be found on page~\pageref{sec:program4}. 

% % % % % % % % % % % % % % % % % % % % % % % % % % % % % % % % % % % % % % % % % % % % % % % % % % % % % % % % % % % % % % % % % % % % % % % % % % % % % % % % % % % % % % % % % % % % % % % % % % % % % % % % % % % % % % % % % % % % % % % % % % % % % % % % % % % % % % % % % % % % % % % % % % % % % % % % % % % % % % % % % % % %

\begin{verbatim}

>> B1 = [ 4 6 5 2 3 ; 5 0 3 0 1 ; 1 5 7 1 0 ; 5 5 0 3 1 ; 2 1 2 4 0 ];
>> B2 = [ 3 7 19 3 9 6 ; 10 2 20 15 3 0 ; 14 1 3 14 11 3 ; 26 1 21 6 3 5 ;
          0 1 3 19 0 3 ];
>> kerBasis(13,B1)

ans =

     7
     2
     1
     2
     1

>> kerBasis(17,B1)

ans =

  5x0 empty double matrix

>> kerBasis(23,B2)

ans =

     6
     6
     9
     9
     9
     1
\end{verbatim}

So $B_1 \pmod {17}$ has trivial kernel. Where $B_1 \pmod {13}$ and $B_2 \pmod {23}$ has basis for the kernel
$\{\begin{pmatrix}
    7  \\
    2 \\
	1 \\
	2 \\
	1 \\ 
   \end{pmatrix}\}$  and
$\{\begin{pmatrix}
    16 \\
    16 \\
	9 \\
	9 \\
	9 \\ 
	1 \\
   \end{pmatrix}\}$ respectively.






\section*{Question 5}

For any matrix $U$, we have $dim(U)+dim(U^{\circ})=\#(\text{rows of }U)$.





\section*{Question 6} 

\begin{verbatim}
>> A1 = [ 0 1 7 2 10 ; 8 0 2 5 1 ; 2 1 2 5 5 ; 7 4 5 3 0];
>> kerBasis(19,A1)

ans =

    13
     6
     3
     1
    18

>> X = kerBasis(19,A1)';
>> kerBasis(19,X)

ans =

    18     9     3    16
    18     0     0     0
     0    18     0     0
     0     0    18     0
     0     0     0    18

>> Y = kerBasis(19,X)';
>> rowechelon(19,Y)

ans =

     1     0     0     0    13
     0     1     0     0     6
     0     0     1     0     3
     0     0     0     1     1

>> rowechelon(19,A1)

ans =

     1     0     0     0    13
     0     1     0     0     6
     0     0     1     0     3
     0     0     0     1     1
\end{verbatim}

We are given that $U$ is the row space of matrix $A_1$. Now \texttt{kerBasis(19,A1)} produces a matrix which has it's columns as a basis for $U^{\circ}$, let it's transpose be $X$. Similarly \texttt{kerBasis(19,X)} produces a matrix which has it's columns as a basis for $(U^{\circ})^{\circ}$, let it's transpose be $Y$. Now we can see that \texttt{rowechelon(19,Y)} and \texttt{rowechelon(19,A1)} give the same matrice, hence they have the same row space. So we can conclude that $U=(U^{\circ})^{\circ}$,






\section*{Question 7}

Programs for computing bases for $U,W,U+W$ and $U\cap W$ can be found on page \pageref{sec:program7}. 

% % % % % % % % % % % % % % % % % % % % % % % % % % % % % % % % % % % % % % % % % % % % % % % % % % % % % % % % % % % % % % % % % % % % % % % % % % % % % % % % % % % % % % % % % % % % % % % % % % % % % % % % % % % % % % % % % % % % % % % % % % % % % % % % % % % % % % % % % % % % % % % % % % % % % % % % % % % % % % % % % % % %

We have $dim(U)+dim(V)=dim(U+W)+dim(U\cap W)$.
\begin{itemize}
\item Modulo $11$ with $U$ the row space of $A1$ and $W$ the row space of $B1$.
\begin{verbatim}
>> A1 = [ 0 1 7 2 10 ; 8 0 2 5 1 ; 2 1 2 5 5 ; 7 4 5 3 0];
>> B1 = [ 4 6 5 2 3 ; 5 0 3 0 1 ; 1 5 7 1 0 ; 5 5 0 3 1 ; 2 1 2 4 0 ];
>> Basis(11,A1)

ans =

     5     4     0     0
     1     0     0     0
     0     1     0     0
     0     0     1     0
     0     0     0     1

>> Basis(11,B1)

ans =

     4     9     6     0
     1     0     0     0
     0     1     0     0
     0     0     1     0
     0     0     0     1

>> sumBasis(11,A1,B1)

ans =

     1     0     0     0     0
     0     1     0     0     0
     0     0     1     0     0
     0     0     0     1     0
     0     0     0     0     1

>> intBasis(11,A1,B1)

ans =

     7     8     0
     5     6     0
     1     0     0
     0     1     0
     0     0     1



\end{verbatim}

\item Modulo $19$ with $U$ the row space of $A3$ and $W$ the kernel of $A3$.

\begin{verbatim}
>> A3 = [ 1 0 0 0 3 0 0 ; 0 5 0 1 6 3 0 ; 0 0 5 0 2 0 0 ; 2 4 0 0 0 5 1 ;
          4 3 0 0 6 2 6 ];
>> kerBasis(19,A3)

ans =

    13    18
    16     5
     3    10
     0    11
     2    13
     1     0
     0     1
>> kerA3 = (kerBasis(19,A3))';

>> Basis(19,A3)

ans =

    10     1    13    18     7
     0    17     0    15     5
     1     0     0     0     0
     0     1     0     0     0
     0     0     1     0     0
     0     0     0     1     0
     0     0     0     0     1

>> Basis(19,kerA3)

ans =

    13    18
    16     5
     3    10
     0    11
     2    13
     1     0
     0     1

>> sumBasis(19,A3,kerA3)

ans =

     1     0     0     0     0     0     0
     0     1     0     0     0     0     0
     0     0     1     0     0     0     0
     0     0     0     1     0     0     0
     0     0     0     0     1     0     0
     0     0     0     0     0     1     0
     0     0     0     0     0     0     1

>> intBasis(19,A3,kerA3)

ans =

  7×0 empty double matrix
\end{verbatim}

\item Modulo $23$ with $U$ the row space of $A3$ and $W$ the kernel of $A3$.

\begin{verbatim}

>> A3 = [ 1 0 0 0 3 0 0 ; 0 5 0 1 6 3 0 ; 0 0 5 0 2 0 0 ; 2 4 0 0 0 5 1 ; 4 3 0 0 6 2 6 ];
>> kerBasis(23,A3)

ans =

    15     1
    20     5
     2    17
    19     0
    18    15
     1     0
     0     1

>> kerA3 = (kerBasis(23,A3))';

>> Basis(23,A3)

ans =

     6    15     8     2    15
     0    20     0    18     6
     1     0     0     0     0
     0     1     0     0     0
     0     0     1     0     0
     0     0     0     1     0
     0     0     0     0     1

>> Basis(23,kerA3)

ans =

    15     1
    20     5
     2    17
    19     0
    18    15
     1     0
     0     1

>> sumBasis(23,A3,kerA3)

ans =

     6     6    10     8     9     2
     1     0     0     0     0     0
     0     1     0     0     0     0
     0     0     1     0     0     0
     0     0     0     1     0     0
     0     0     0     0     1     0
     0     0     0     0     0     1

>> intBasis(23,A3,kerA3)

ans =

    11
     3
     3
     5
     4
    16
     1

\end{verbatim}

\end{itemize}



\section*{Question 8}

If we had the field of real numbers instead of $GF(23)$, then $U\cap W$ would be the empty space. Becasue $\forall (x,y) \in (U,W)$, we have $x.y = 0$, so if there exists $v\in U\cap W$, we must have $v.v = 0$. But that's possible if and only if $v=0$, hence $U\cap W = 0$. This is not the case when we work over $GF(23)$ as it's not a ordered field, so the inner product doesn't define a norm. The last example in Question 7 gives a contradiction for that.




\newpage
\section*{Program for Question 1}
\label{sec:program1}
\begin{verbatim}
(i) function [I] = inverse (p)
for i=1:p-1
    for j=1:p-1
        if mod(i*j,p) == 1     
            I(i) = j;         
            break         
        else         
            j = j+1; 
        end
    end
end
end

(ii) function [I] = inverse2 (p)

for i=1:(p-1)/2
    for j=1:p-1
        if mod(i*j,p) == 1     
            I(i) = j;
            I(p-i) = p-j;
            break         
        else         
            j = j+1; 
        end
    end
end

end
\end{verbatim}







\newpage
\section*{Program for Question 3}
\label{sec:program3}
\begin{verbatim}
function [A,l] = rowechelon(p,A)

I = inverse(p);
A = mod(A,p);
[m,n]=size(A);
l = zeros(1,0);
L = 1;

for i = 1:n
    for j = L:m
        if A(j,i) ~= 0
            A(j,:) = mod( A(j,:).*I(A(j,i)) ,p); 
            
            storeMe = A(j,:); % store row v of A
            A(j,:) = A(L,:); % copy row u of A into row v of A
            A(L,:) = storeMe; % copy the stored row into row u of A
            
            for k = 1:m 
                if k == L
                    A=A;
                else
                A(k,:) = mod( A(k,:) - A(k,i).*A(L,:) ,p);
                end
            end
            l(L) = i;
            L = L+1;
            
            break
           
        end
    end
end

end
\end{verbatim} 








\newpage
\section*{Program for Question 4}
\label{sec:program4}
\begin{verbatim}
function [C] = kerBasis(p,A)

[m,n]=size(A);
[A,l] = rowechelon(p,A);
r = length(l);
q = setdiff(1:n,l);
   
C = zeros(n,n-r);   
for i = 1:n-r    
     for k = 1:r
        C(l(k),i) = mod(-A(k,q(i)),p);    
     end
        C(q(i),i) = 1;
end  

end
\end{verbatim}









\newpage
\section*{Program for Question 7}
\label{sec:program7}
\begin{verbatim}
(i) Basis.m
function [U] = Basis(p,A)

C = kerBasis(p,A);
U = kerBasis(p,C.');

end

(ii) sumBasis.m
function [Z] = sumBasis(p,A,B)

U = Basis(p,A);
W = Basis(p,B);

Y = union(U.',W.','rows');
Z = Basis(p,Y);

end

(iii) intBasis.m
function [R] = intBasis(p,A,B)

C = kerBasis(p,A);
D = kerBasis(p,B);

Q = sumBasis(p,C.',D.');
R = kerBasis(p,Q.');

end
\end{verbatim}
\end{document}  